\documentclass[10pt,a4paper]{article}

\usepackage{amssymb}
\usepackage[french,english]{babel}
\usepackage[utf8]{inputenc}
\usepackage{graphics}
\usepackage{lineno}
\usepackage{cite}
\usepackage{float}
\usepackage{ccaption}
\usepackage{caption}
\usepackage{array}
\usepackage{lscape}
\usepackage[hmargin=2cm,vmargin=2cm]{geometry}
\usepackage{fancyvrb}

\title{\bf \Large A short manual for {\tt sNMF}\\
\large (command-line version)
}

\author{
        Eric Frichot\\efrichot@gmail.com\\
}

\newcommand{\bp}{\mathbf{p}}
\newcommand{\LLL}{\mathcal{L}}

%% BEGIN DOC
\begin{document}

\maketitle
\begin{center}
{\it Please, print this reference manual only if it is necessary.}
\end{center}

\noindent
This short manual aims to help users to run {\tt sNMF} command-line engine on Mac and Linux. 

\section{Description} 
Inference of individual admixture coefficients, which is important for population genetic and association studies, is commonly performed using compute-intensive likelihood algorithms. With the availability of large population genomic data sets, fast versions of likelihood algorithms have attracted considerable attention. Reducing the computational burden of estimation algorithms remains, however, a major challenge. Here, we present a fast and efficient method for estimating individual admixture coefficients based on sparse non-negative matrix factorization algorithms. We implemented our method in the computer program {\tt sNMF}, and applied it to human and plant genomic data sets. The performances of {\tt sNMF} were then compared to the likelihood algorithm implemented in the computer program {\tt ADMIXTURE}.  Without loss of accuracy, {\tt sNMF} computed estimates of admixture coefficients within run-times approximately 10 to 30 times faster than those of {\tt ADMIXTURE}
\\
\\
\noindent
Eric Frichot, François Mathieu, Théo Trouillon, Guillaume Bouchard, Olivier François. {\it Fast Inference of Admixture Coefficients Using Sparse Non-negative Matrix Factorization Algorithms}, submitted. 

\section{Installation} 

\noindent
To install {\tt sNMF} command-line version, unzip the sNMF\_CL.zip file, and run the 
install script (install.command) from the {\tt sNMF} directory.
From a terminal shell, go to {\tt sNMF} main directory and type "./install.command".
If the script is not executable, type "chmod +x install.command" and then "./install.command".
A set of binaries should be created in {\tt sNMF} directory.

\section{Data format}

The {\tt sNMF} input file consists of a single genotype file. 
\\
\\
geno (example.geno)\\
The {\bf genotype file} format has one row for each SNP.
  Each row contains 1 character per individual:
  0 means zero copies of reference allele.
  1 means one copy of reference allele.
  2 means two copies of reference allele.
  9 means missing data.

Below, an example of genotype file for $n=3$ individuals and $L=5$ loci.
\begin{center}
\footnotesize
\begin{Verbatim}[frame=single]
112
010
091
121
\end{Verbatim}
\end{center}

\noindent
There are 2 {\bf output files}.

\begin{itemize}
\item The first file (with extension {\bf .Q}) contains individual admixture coefficients.
It is a matrix with $n$ rows (the number of individuals) and $K$ columns (the 
number of ancestral populations).
\item The second file (with extension {\bf .F}) contains the ancestral genotype frequencies.
It is a matrix with $nc\times L$ lines (the number of alleles times the number of SNPs) and $K$ columns (the 
number of ancestral populations). For each SNP, the first line contains the ancestral frequencies for allele 0, the second line for allele 1, ... .
\end{itemize}

\section{Run the programs}
The program is executed from a command line. The format is:
\begin{Verbatim}[frame=single]
./sNMF -g genotype_file.geno -K number_of_ancestral_populations 
\end{Verbatim}

\noindent
All options are mandatory. There is no order for the options in the command line. 
Here is a description of the options:
\begin{itemize}
\item \verb|-g genotype_file.geno| is the path for the genotype file (in .geno format).
\item \verb|-K number_of_ancestral_populations| is the number of ancestral populations. 
\end{itemize}

\noindent
Additional options are available:
\begin{itemize}
\item \verb|-p p| is the number of processes that you choose to use if you run the algorithm in 
a parrallel computer. Be aware that the number of process has to be lower or equal than the number 
of physical processes available on your computer (default: 1).
\item \verb|-i iteration_number| is the max number of iterations in algorithm (default: 200). 
The algorithm should not go until the max number of iterations. The stopping criterion should 
depend on the tolerance error only.
\item \verb|-a alpha| is the value of the regularization parameter (by default: 101). Results can depend on the value of this parameter, especially for small data sets. 
\item \verb|-e tolerance| is the tolerance error (by default: 0.0001). 
\item \verb|-s seed| is the initialization for the random parameter (by default: random). 
\item \verb|-m ploidy|  1 if haploid, 2 if diploid (default: 2). 
\end{itemize}


\noindent
If you need a summary of the options, you can use the \verb|-h| option by typing the command line
\footnotesize
\begin{Verbatim}[frame=single]
./sNMF -h
\end{Verbatim}
\noindent
\normalsize

\noindent
A full example is available at the end of this note.

\section{Cross-Entropy criterion}

We provide two programs that compute a cross entropy score for the data.
\begin{itemize}
\item A first program creates a data set with a given percentage of missing data from your original data set.
The command line format is:
\begin{Verbatim}[frame=single]
./createDataSet -g genotype_file.geno
\end{Verbatim}

The mandatory option is:
\begin{itemize}
\item \verb|-g genotype_file.geno| is the path for the genotype file (in .geno format).
\end{itemize}

It will create a file with around 5 \% of masked data with the name genotype\_file{\bf\_I}.geno with a {\bf\_I} extension to differentiate this file from the original file. 

\noindent
Other options (not mandatory):
\begin{itemize}
\item \verb|-r percentage| is the percentage of masked data in your data set (default: 0.05). 
\item \verb|-e tolerance| is the tolerance error (by default: 0.0001). 
\item \verb|-s seed| is the initialization for the random parameter (by default: random). 
\item \verb|-m ploidy| is 1 if haploid, 2 if diploid (default: 2). 
\end{itemize}

\item A second program calculates the cross-entropy criterion for all data and for the masked data from the 
output of {\tt sNMF}. The cross-entropy criterion is useful to choose the best run for numbers of 
ancestral distinct populations ($K$) and different values of the regularization parameter ($\alpha$). 
A smaller value of cross-entropy with missing data means a better prediction of the data.
The command line format is:
\begin{Verbatim}[frame=single]
./crossEntropy -g genotype_file.geno -K number_of_ancestral_populations
\end{Verbatim}

The mandatory options are:
\begin{itemize}
\item \verb|-g genotype_file.geno| is the path for the genotype file (in .geno format).
\item \verb|-K number_of_ancestral_populations| is the number of $K$ of ancestral populations.
\end{itemize}

In this case, the output from {\tt sNMF}, the files with masked data, the original files and the results files 
are stored in the same directory.

\noindent
Other option (are not mandatory):
\begin{itemize}
\item \verb|-m ploidy|  1 if haploid, 2 if diploid (default: 2). 
\end{itemize}

\end{itemize}

\section{Tutorial}

\subsection{Data set}
The data set that we analyze in this tutorial is an Asian human data set of SNPs data.
This data is a worldwide sample of genomic DNA (10757 SNPs) from 934 individuals,
taken from the Harvard Human Genome Diversity Project - Centre
Etude Polymorphism Humain (Harvard HGDP-CEPH)2 . 
In those data, each marker has been ascertained in samples of Mongolian
ancestry (referenced population HGDP01224) \cite{Patterson_2012}. 

\subsection{Create a data set with masked data}

In the main directory, type:
\begin{Verbatim}[frame=single]
./createDataSet -g examples/panel11.geno
\end{Verbatim}

\noindent
A file with 5 \% of masked data with path \verb|examples/panel11_I.geno| has been created.

\subsection{Run {\tt sNMF}}

Then, run {\tt sNMF} for the data set with 5 \% of masked data (with $K=5$ for example):
\begin{Verbatim}[frame=single]
./sNMF -g examples/panel11_I.geno -K 5
\end{Verbatim}

\noindent
The results files \verb|examples/panel11_I.Q| \verb|examples/panel11_I.F| have been created.

\subsection{Compute the Cross-Entropy criterion}

Finally, calculate the cross-entropy criterion:
\begin{Verbatim}[frame=single]
./crossEntropy -g examples/panel_11.geno -K 5
\end{Verbatim}

\noindent
With this procedure, we can compute a value for the cross-entropy criterion for each of your analysis. 
It is a way to  choose the best run for different number of ancestral populations ($K$) and for 
different values of the regularization parameter ($\alpha$).

\section{Contact}
If you need assistance, do not hesitate to send me an email (efrichot@gmail.com or eric.frichot@imag.fr). 
A FAQ (Frequently Asked Questions) section is available 
on our webpage (ttp://membres-timc.imag.fr/Olivier.Francois/snmf.html). 
{\tt sNMF} software is still under development. All your comments and feedbacks are more than welcome.

\bibliography{biblio}
\bibliographystyle{plain}

\end{document}
